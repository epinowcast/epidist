\documentclass[12pt]{article}
% \usepackage[top=1in,left=1in, right = 1in, footskip=1in]{geometry}
\usepackage[top=1in,footskip=1in]{geometry}

\usepackage{graphicx}
\usepackage{xspace}
%\usepackage{adjustbox}

\usepackage{grffile}

\newcommand{\comment}{\showcomment}
%% \newcommand{\comment}{\nocomment}

\newcommand{\showcomment}[3]{\textcolor{#1}{\textbf{[#2: }\textsl{#3}\textbf{]}}}
\newcommand{\nocomment}[3]{}

\newcommand{\jd}[1]{\comment{cyan}{JD}{#1}}
\newcommand{\swp}[1]{\comment{magenta}{SWP}{#1}}
\newcommand{\bmb}[1]{\comment{blue}{BMB}{#1}}
\newcommand{\djde}[1]{\comment{red}{DJDE}{#1}}

\newcommand{\eref}[1]{Eq.~(\ref{eq:#1})}
\newcommand{\fref}[1]{Fig.~\ref{fig:#1}}
\newcommand{\Fref}[1]{Fig.~\ref{fig:#1}}
\newcommand{\sref}[1]{Sec.~\ref{#1}}
\newcommand{\frange}[2]{Fig.~\ref{fig:#1}--\ref{fig:#2}}
\newcommand{\tref}[1]{Table~\ref{tab:#1}}
\newcommand{\tlab}[1]{\label{tab:#1}}
\newcommand{\seminar}{SE\mbox{$^m$}I\mbox{$^n$}R}

\usepackage{amsthm}
\usepackage{amsmath}
\usepackage{amssymb}
\usepackage{amsfonts}
\usepackage[utf8]{inputenc} % make sure fancy dashes etc. don't get dropped

\usepackage{lineno}
\linenumbers

\usepackage[pdfencoding=auto, psdextra]{hyperref}

\usepackage{natbib}
\bibliographystyle{unsrt}
\date{\today}

\usepackage{xspace}
\newcommand*{\ie}{i.e.\@\xspace}

\usepackage{color}

\newcommand{\Rx}[1]{\ensuremath{{\mathcal R}_{#1}}\xspace} 
\newcommand{\RR}{\ensuremath{{\mathcal R}}\xspace}
\newcommand{\Rres}{\Rx{\mathrm{res}}}
\newcommand{\Rinv}{\Rx{\mathrm{inv}}}
\newcommand{\Rhat}{\ensuremath{{\hat\RR}}}
\newcommand{\Rt}{\ensuremath{{\mathcal R}(t)}\xspace}
\newcommand{\tsub}[2]{#1_{{\textrm{\tiny #2}}}}
\newcommand{\dd}[1]{\ensuremath{\, \mathrm{d}#1}}
\newcommand{\dtau}{\dd{\tau}}
\newcommand{\dx}{\dd{x}}
\newcommand{\dsigma}{\dd{\sigma}}

\newcommand{\rx}[1]{\ensuremath{{r}_{#1}}\xspace} 
\newcommand{\rres}{\rx{\mathrm{res}}}
\newcommand{\rinv}{\rx{\mathrm{inv}}}

\newcommand{\psymp}{\ensuremath{p}} %% primary symptom time
\newcommand{\ssymp}{\ensuremath{s}} %% secondary symptom time
\newcommand{\pinf}{\ensuremath{\alpha_1}} %% primary infection time
\newcommand{\sinf}{\ensuremath{\alpha_2}} %% secondary infection time

\newcommand{\psize}{{\mathcal P}} %% primary cohort size
\newcommand{\ssize}{{\mathcal S}} %% secondary cohort size

\newcommand{\gtime}{\tau_{\rm g}} %% generation interval
\newcommand{\gdist}{g} %% generation-interval distribution
\newcommand{\idist}{\ell} %% incubation-period distribution

\newcommand{\total}{{\mathcal T}} %% total number of serial intervals

\newcommand{\tend}{{t_{\mathrm{end}}}}
\newcommand{\tmin}{{t_{\mathrm{max}}}}
\newcommand{\tmax}{{t_{\mathrm{max}}}}
\newcommand{\trep}{{t_{\mathrm{rep}}}}
\newcommand{\pmin}{{p_{\mathrm{min}}}}
\newcommand{\pmax}{{p_{\mathrm{max}}}}
\newcommand{\smin}{{s_{\mathrm{min}}}}
\newcommand{\smax}{{s_{\mathrm{max}}}}

\usepackage{lettrine}

\newcommand{\dropcapfont}{\fontfamily{lmss}\bfseries\fontsize{26pt}{28pt}\selectfont}
\newcommand{\dropcap}[1]{\lettrine[lines=2,lraise=0.05,findent=0.1em, nindent=0em]{{\dropcapfont{#1}}}{}}

\begin{document}

\begin{flushleft}{
	\Large
	\textbf\newline{
		Adjusting for common biases in infectious disease data when estimating epidemiological delay distributions
	}
}
\newline
\\
Authors
\bigskip

\bigskip

\section*{Abstract}

\end{flushleft}

\pagebreak

\section{Introduction}

\swp{Maybe we don't have to define what a delay distribution is. But I think we need to talk about why these delays are useful and why we should care.}
Epidemiological delay distributions (i.e., the distribution of time between two events associated with infection) provide useful means of summarizing the course of infection, such as disease progression \citep{lauer2020incubation,verity2020estimates} and temporal variation in infectiousness.
\swp{A sentence about using delays to measure NPIS?}
Accurate characterization of epidemiological delay distributions are also critical to understanding population-level outbreak dynamics. \swp{maybe need examples or not.}

Estimating epidemiological delay distributions from data can be challenging due to observational biases.
For example, when the epidemic is ongoing, we cannot observe events that have not happened yet---this \emph{truncation bias}, in turn, causes us to observe shorter intervals and underestimate the mean of the focal distribution.
In addition, event times are often reported in \emph{censored} intervals that can span over a day or even several weeks;
inaccurate estimates of event times can further translate to biases in the estimates of delay distributions.
Recent studies have also highlighted the role of dynamical biases ...

\section{General theory for understanding epidemiological delay distributions}

\swp{I think we need to build up some theory before we introduce actual methods?}

\subsection{Notation}

Throughout this paper, we use $p$ and $s$ to denote primary and secondary event times, where $s-p$ is the focal epidemiological delay.
In particular we are interested in estimating the forward distribution $f_p(\tau)$, measured from the cohort of individuals who experienced the primary event at the same time $p$.
For now, we focus on the scenario in which the forward distribution remain stable over the course of an epidemic, $f_p(\tau) = f(\tau)$---this is true for many distributions that reflect the course of infection, such as the incubation-period distribution (i.e., time between infection and symptom onset).
Other distributions, such as the generation-interval distribution (i.e., time between infection and transmission) and reporting distributions, are expected to change throughout.
While it is possible (and perhaps more convenient) to characterize the backward distribution $b_s(\tau)$, measured from the cohort of individuals who experienced the secondary event at the same time $s$, these distributions are subject to dynamical biases as we discuss later.

\subsection{Truncation bias}

When an epidemic is ongoing, we cannot observe events that have not happened yet.
Therefore, in order to observe ...
The truncated distribution then follows:
\begin{equation}
\frac{f_p(\tau)}{F_p(\tend-p) }
\end{equation}

\subsection{Censoring bias}

There are broadly two kinds of censoring in epidemic data: explicit and implicit.
Explicit censoring refers to a scenario in which a focal event is recorded as having occurred within a censored interval between $\tmin$ and $\tmax$.
For example, infection events are often explicitly censored because the exact timing of infection is difficult to identify.
Implicit censoring refers to a scenario in which a focal event is recorded as having occurred at a specific time point $\trep$ but implicitly has an associated .
For example, an individual who developed symptoms on a given day could have started showing symptoms at any time within the 24 hour cycle.


\section{Methods for inferring epidemiological delay distributions from data}

\section{Simulation study}

\section{Case study with real data}

\section{Discussion}

\pagebreak

\bibliography{dynamicaltruncation}

\end{document}
