\documentclass[12pt]{article}
% \usepackage[top=1in,left=1in, right = 1in, footskip=1in]{geometry}
\usepackage[top=1in,footskip=1in]{geometry}

\usepackage{graphicx}
\usepackage{xspace}
%\usepackage{adjustbox}

\usepackage{grffile}

\newcommand{\comment}{\showcomment}
%% \newcommand{\comment}{\nocomment}

\newcommand{\showcomment}[3]{\textcolor{#1}{\textbf{[#2: }\textsl{#3}\textbf{]}}}
\newcommand{\nocomment}[3]{}

\newcommand{\jd}[1]{\comment{cyan}{JD}{#1}}
\newcommand{\swp}[1]{\comment{magenta}{SWP}{#1}}
\newcommand{\bmb}[1]{\comment{blue}{BMB}{#1}}
\newcommand{\djde}[1]{\comment{red}{DJDE}{#1}}

\newcommand{\eref}[1]{Eq.~(\ref{eq:#1})}
\newcommand{\fref}[1]{Fig.~\ref{fig:#1}}
\newcommand{\Fref}[1]{Fig.~\ref{fig:#1}}
\newcommand{\sref}[1]{Sec.~\ref{#1}}
\newcommand{\frange}[2]{Fig.~\ref{fig:#1}--\ref{fig:#2}}
\newcommand{\tref}[1]{Table~\ref{tab:#1}}
\newcommand{\tlab}[1]{\label{tab:#1}}
\newcommand{\seminar}{SE\mbox{$^m$}I\mbox{$^n$}R}

\usepackage{amsthm}
\usepackage{amsmath}
\usepackage{amssymb}
\usepackage{amsfonts}
\usepackage[utf8]{inputenc} % make sure fancy dashes etc. don't get dropped

\usepackage{lineno}
\linenumbers

\usepackage[pdfencoding=auto, psdextra]{hyperref}

\usepackage{natbib}
\bibliographystyle{chicago}
\date{\today}

\usepackage{xspace}
\newcommand*{\ie}{i.e.\@\xspace}

\usepackage{color}

\newcommand{\Rx}[1]{\ensuremath{{\mathcal R}_{#1}}\xspace} 
\newcommand{\RR}{\ensuremath{{\mathcal R}}\xspace}
\newcommand{\Rres}{\Rx{\mathrm{res}}}
\newcommand{\Rinv}{\Rx{\mathrm{inv}}}
\newcommand{\Rhat}{\ensuremath{{\hat\RR}}}
\newcommand{\Rt}{\ensuremath{{\mathcal R}(t)}\xspace}
\newcommand{\tsub}[2]{#1_{{\textrm{\tiny #2}}}}
\newcommand{\dd}[1]{\ensuremath{\, \mathrm{d}#1}}
\newcommand{\dtau}{\dd{\tau}}
\newcommand{\dx}{\dd{x}}
\newcommand{\dsigma}{\dd{\sigma}}

\newcommand{\rx}[1]{\ensuremath{{r}_{#1}}\xspace} 
\newcommand{\rres}{\rx{\mathrm{res}}}
\newcommand{\rinv}{\rx{\mathrm{inv}}}

\newcommand{\psymp}{\ensuremath{p}} %% primary symptom time
\newcommand{\ssymp}{\ensuremath{s}} %% secondary symptom time
\newcommand{\pinf}{\ensuremath{\alpha_1}} %% primary infection time
\newcommand{\sinf}{\ensuremath{\alpha_2}} %% secondary infection time

\newcommand{\psize}{{\mathcal P}} %% primary cohort size
\newcommand{\ssize}{{\mathcal S}} %% secondary cohort size

\newcommand{\gtime}{\tau_{\rm g}} %% generation interval
\newcommand{\gdist}{g} %% generation-interval distribution
\newcommand{\idist}{\ell} %% incubation-period distribution

\newcommand{\total}{{\mathcal T}} %% total number of serial intervals

\newcommand{\tend}{{t_{\mathrm{end}}}}
\newcommand{\tmin}{{t_{\mathrm{min}}}}
\newcommand{\tmax}{{t_{\mathrm{max}}}}
\newcommand{\trep}{{t_{\mathrm{rep}}}}
\newcommand{\pmin}{{p_{\mathrm{min}}}}
\newcommand{\pmax}{{p_{\mathrm{max}}}}
\newcommand{\prep}{{p_{\mathrm{rep}}}}
\newcommand{\smin}{{s_{\mathrm{min}}}}
\newcommand{\smax}{{s_{\mathrm{max}}}}
\newcommand{\srep}{{s_{\mathrm{rep}}}}

\usepackage{lettrine}

\newcommand{\dropcapfont}{\fontfamily{lmss}\bfseries\fontsize{26pt}{28pt}\selectfont}
\newcommand{\dropcap}[1]{\lettrine[lines=2,lraise=0.05,findent=0.1em, nindent=0em]{{\dropcapfont{#1}}}{}}

\begin{document}

\begin{flushleft}{
	\Large
	\textbf\newline{
		Adjusting for common biases in infectious disease data when estimating epidemiological delay distributions
	}
}
\newline
\\
Authors
\bigskip

\bigskip

\section*{Abstract}

\end{flushleft}

\pagebreak

\section{Introduction}

\swp{Maybe we don't have to define what a delay distribution is. But I think we need to talk about why these delays are useful and why we should care.}
Epidemiological delay distributions (i.e., the distribution of time between two events associated with infection) provide useful means of summarizing the course of infection, such as disease progression \citep{lauer2020incubation,verity2020estimates} and temporal variation in infectiousness \citep{madewell2022serial}.
\swp{A sentence about using delays to measure NPIS?}
Accurate characterization of epidemiological delay distributions are also critical to understanding population-level outbreak dynamics. \swp{maybe need examples or not.}

However, epidemiological delays can be difficult to measure in many cases because focal events cannot be observed directly or may be known imprecisely.
In these cases, event times are often reported in \emph{censored} intervals that can span over a day or even several weeks.
In addition, when an epidemic is ongoing, epidemiological delay distributions can be \emph{truncated} because we cannot observe events that have not happened yet.
Although both censoring and truncation problems have been widely discussed in epidemiology and other literatures, there are remaining gaps.
For example, most methods assume the the true event time is uniformly distributed within the censored interval;
but as we show later, the distribution of event time depends on epidemiological dynamics.
Other studies neglect the possibility that censoring of event times can also affect the truncation problem.

Recent studies have also highlighted the role of dynamical biases ...

\swp{A paragraph on what's been done. Some people have focused on truncation. Some people on dynamical bias. Some people tried to correct for both. Censoring is also often under-appreciated, at least at a daily time step.}

\section{General theory for estimating epidemiological delay distributions}

\swp{I started writing this section thinking that we need to build up some theory before we introduce actual methods. But now I feel like we can combine this section with the methods section to derive some theory and show likelihood...if that makes sense...}
We present general theory for understanding observational biases in estimating epidemiological delay distributions and derive likelihood for each scenario.
We also provide approximations of each methods, which we compare later.

\subsection{Notation}

Throughout this paper, we use $p$ and $s$ to denote primary and secondary event times, where $s-p$ is the focal epidemiological delay.
In particular we are interested in estimating the forward distribution $f_p(\tau)$, measured from the cohort of individuals who experienced the primary event at the same time $p$.
For now, we focus on the scenario in which the forward distribution remain stable over the course of an epidemic, $f_p(\tau) = f(\tau)$---this is true for many distributions that reflect the course of infection, such as the incubation-period distribution (i.e., time between infection and symptom onset).
Other distributions, such as the generation-interval distribution (i.e., time between infection and transmission) and reporting distributions, are expected to change throughout.
While it is possible (and perhaps more convenient) to characterize the backward distribution $b_s(\tau)$, measured from the cohort of individuals who experienced the secondary event at the same time $s$, these distributions are subject to dynamical biases as we discuss later.

\subsection{Accounting for truncation bias}

When an epidemic is ongoing, we cannot observe events that have not happened yet.
Therefore, in order to observe ...
Then, the likelihood of observing a delay of $X_i$ from the right-truncated distribution is given by:
\begin{equation}
\mathcal L(\theta|X_i) = \frac{f(X_i)}{F(\tend-p)}.
\end{equation}

\subsection{Accounting for censoring bias}

There are broadly two kinds of censoring in epidemic data: explicit and implicit.
Explicit censoring refers to a scenario in which a focal event is reported as having occurred within a censored interval between $\tmin$ and $\tmax$.
For example, infection events are often explicitly censored because the exact timing of infection is difficult to identify.
Implicit censoring refers to a scenario in which focal events that occur between some interval are reported as having occurred at a specific time point $\trep$.
For example, an individual who developed symptoms on a given day could have started showing symptoms at any time within the 24 hour cycle.
\swp{would be good to mention that people often ignore the implicit censoring}
In this paper, we will be primarily dealing with censoring that occurs at a daily time step and therefore will focus on implicit censoring---the implicit censoring also allows for a convenient framing of the problem as we will show here.

Here, we take the cohort-based approach to understanding censored data.
We begin by assuming that an individual who experiences a focal event at time $\tmin leq t < \tmax$ has a probability $g_t(\trep)$ of having their event reported at time $\trep$.
Then, for a cohort of individuals who reported the focal event at time $\trep$, the distribution of their true event time $\pi(t)$ depend on the reporting distribution $g_t(\trep)$ as well as the incidence of focal event $i(t)$:
\begin{equation}
\pi(t) = \frac{i(t) g_t(\trep)}{\int_{\tmin}^\tmax i(z) g_z(\trep) \dd z}.
\end{equation}
For example, for an idealized daily reporting scenario, in which all events happening at time $n \leq t < n+1$ is reported as having occurred at time $n$, the reporting distribution corresponds to $g_t(n) = 1$ for $n \leq t < n+1$;
in this case, if we further assume that the incidence is exponentially changing at rate $r$, the distribution of their true event time simplifies to:
\begin{equation}
\pi(t) = \frac{\exp(rt)}{\exp(nr) (\exp(r) - 1)/r}.
\end{equation}
We note that the denominator can be ignored in the likelihood for the purpose of parameter estimation because it is a constant value.

Finally, given that primary and secondary events are reported at time $P_i$ and $S_i$, respectively, the likelihood of this data is given by:
\begin{equation}
\mathcal L(\theta|P_i, S_i) = \int_{S_i}^{S_i+1} \int_{P_i}^{P_i+1} \pi_s(t_2) \pi_p(t_1) f(t_2-t_1) \dd t_1 \dd t_2,
\end{equation}
where $t_1$ and $t_2$ represent latent variables for the timing of primary and secondary events, respectively;
and $\pi_p$ and $\pi_s$ represent distributions of the true timing of primary and secondary events, respectively.
Other studies have typically assumed uniform distributions for $\pi$, thereby neglecting underlying epidemioligical dynamics.
\swp{Need to say we're going to compare uniform vs exponential?}

\subsection{Accounting for both truncation and censoring bias}

Correct method:
\begin{equation}
\mathcal L(\theta|P_i, S_i) = \int_{S_i}^{S_i+1} \int_{P_i}^{P_i+1} \pi_s(t_2) \pi_p(t_1) \frac{f(t_2 - t_1)}{F_p(\tend-t_1)} \dd t_1 \dd t_2
\end{equation}

The correct method is complex because we have to rely on latent variables. In particular, the truncation time now depends on censoring. Instead, we can approximate using half-points and what not...

\section{Implementation}

\begin{itemize}
  \item Let's define likelihood with theory in the previous section
  \item Primarily focus on actual implementation here. Stan, brms, blah blah blah
\end{itemize}

\section{Simulation study}

\subsection{Comparing censoring and truncation methods}

\subsection{Discrete- vs continuous-time distributions}

\subsection{Dynamical correction}

\section{Case study with real data}

\section{Discussion}

\begin{itemize}
	\item Longer or more complex censoring
	\item Joint estimation of the transmission process and distributions
	\item Time-varying distributions
\end{itemize}	

\pagebreak

\bibliography{dynamicaltruncation}

\end{document}
